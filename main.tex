\documentclass[
	openany,
	% -- opções da classe memoir --
	12pt,				% tamanho da fonte
   % openright,
	%twoside,
    oneside,
    % para impressão em verso e anverso. Oposto a oneside
	a4paper,			% tamanho do papel. 
	brazil				% o último idioma é o principal do documento
	]{abntex2}

% ---
% Pacotes básicos 
% ---

\usepackage{fontspec}
\setmainfont{Arial}

\usepackage[utf8]{inputenc}		% Codificacao do documento (conversão automática dos acentos)
\usepackage{indentfirst}		% Indenta o primeiro parágrafo de cada seção.
\usepackage{color}				% Controle das cores
\usepackage{graphicx}			% Inclusão de gráficos
\usepackage{microtype} 			% para melhorias de justificação
\usepackage{multicol}			% multiplas colunas no texto
\usepackage{subcaption}
\usepackage{caption}
\usepackage{float}
\usepackage{amsmath}
\usepackage{amssymb}
\usepackage{amsthm}
\usepackage{lipsum}
\usepackage{blindtext}


% ---
% ---
% Pacotes de citações
% ---
\usepackage[brazilian]{backref}	 % Paginas com as citações na bibl
\usepackage[alf]{abntex2cite}	% Citações padrão ABNT

% --- 
% CONFIGURAÇÕES DE PACOTES
% --- 

% ---
% Configurações do pacote backref
% Usado sem a opção hyperpageref de backref
\renewcommand{\backrefpagesname}{Citado na(s) página(s):~}
% Texto padrão antes do número das páginas
\renewcommand{\backref}{}
% Define os textos da citação
\renewcommand*{\backrefalt}[4]{
	\ifcase #1 %
		Nenhuma citação no texto.%
	\or
		Citado na página #2.%
	\else
		Citado #1 vezes nas páginas #2.%
	\fi}%
% ---

% ---
% Informações de dados para CAPA e FOLHA DE ROSTO
% ---
\titulo{Aplicativo móvel de apoio à gestão de eventos acadêmicos}
\autor{Luidi Matheus Silva de Oliveira\\Luiz Fellipe Saidler Leitier}
\local{Campos dos Goytacazes}
\data{2021}
\orientador{Fernando Luiz de Carvalho e Silva}
\coorientador{Ronaldo Amaral Santos}

\instituicao{%
	Instituto Federal Fluminense campus Campos Centro
    \par
  	Curso de Bacharelado em Sistemas de Informação
}

\tipotrabalho{Trabalho de Conclusão de Curso}
% O preambulo deve conter o tipo do trabalho, o objetivo, 
% o nome da instituição e a área de concentração 
\preambulo{Monografia apresentada ao Curso de Bacharelado em Sistemas de Informação do Instituto Federal Fluminense campus Campos Centro, como requisito parcial para obtenção do título de Bacharel em Sistemas de Informação.}
% ---


% ---
% Configurações de aparência do PDF final

% alterando o aspecto da cor azul
\definecolor{blue}{RGB}{41,5,195}

% informações do PDF
\makeatletter
\hypersetup{
     	%pagebackref=true,
		pdftitle={\@title}, 
		pdfauthor={\@author},
    	pdfsubject={\imprimirpreambulo},
	    pdfcreator={LaTeX with abnTeX2},
		colorlinks=true,       		% false: boxed links; true: colored links
    	linkcolor=blue,          	% color of internal links
    	citecolor=blue,        		% color of links to bibliography
    	filecolor=magenta,      		% color of file links
		urlcolor=blue,
		bookmarksdepth=4
}
\makeatother
% --- 

% --- 
% Espaçamentos entre linhas e parágrafos 
% --- 

% O tamanho do parágrafo é dado por:
\setlength{\parindent}{1.3cm}

% Controle do espaçamento entre um parágrafo e outro:
\setlength{\parskip}{0.2cm}  % tente também \onelineskip

% ---
% compila o indice
% ---
\makeindex
% ---

% ----
% Início do documento
% ----
\begin{document}

% Seleciona o idioma do documento (conforme pacotes do babel)
%\selectlanguage{english}
\selectlanguage{brazil}

% Retira espaço extra obsoleto entre as frases.
\frenchspacing 

% ----------------------------------------------------------
% ELEMENTOS PRÉ-TEXTUAIS
% ----------------------------------------------------------
% \pretextual
%\begin{figure}[h]
%\centering % este comando é usado para centralizar a figura
%\includegraphics[width=7cm]{figuras/logo_ufrpe_horizontal.png}\\
%\end{figure}

% \begin{figure}[ht]
% \centering
% \begin{minipage}[b]{0.45\textwidth}
% \includegraphics[height=3cm]{figuras/logo_ufrpe_horizontal.png}
% \end{minipage}
% \qquad
% \begin{minipage}[b]{0.45\textwidth}
% \includegraphics[height=2.5cm]{figuras/logo_bsi.pdf}
% \end{minipage}
% \end{figure}

%\begin{minipage}[t]{1\textwidth}
% 	\begin{figure}[ht]
% 		\includegraphics[height=3cm]{figuras/logo_ufrpe_horizontal.png}
% 		\hspace{4.5cm}
%     	\includegraphics[height=2.5cm]{figuras/logo_bsi.pdf}
% 	\end{figure}    
%\end{minipage}

% ---
% Capa
% ---
\imprimircapa
% ---
% ---
% Folha de rosto
% (o * indica que haverá a ficha bibliográfica)
% ---
\imprimirfolhaderosto

% Inserir folha de aprovação
% ---

% Isto é um exemplo de Folha de aprovação, elemento obrigatório da NBR
% 14724/2011 (seção 4.2.1.3). Você pode utilizar este modelo até a aprovação
% do trabalho. Após isso, substitua todo o conteúdo deste arquivo por uma
% imagem da página assinada pela banca com o comando abaixo:
%
% \includepdf{folhadeaprovacao_final.pdf}
%
\begin{folhadeaprovacao}

  \begin{center}
    {\ABNTEXchapterfont\large\imprimirautor}

    \vspace*{\fill}\vspace*{\fill}
    \begin{center}
      \ABNTEXchapterfont\bfseries\Large\imprimirtitulo
    \end{center}
    \vspace*{\fill}
    
    \hspace{.45\textwidth}
    \begin{minipage}{.5\textwidth}
        \imprimirpreambulo
    \end{minipage}%
    \vspace*{\fill}
   \end{center}
        
   Trabalho aprovado.
   \imprimirlocal, 09 de Setembro de 2021:

   \assinatura{\textbf{Fernando Luiz de Carvalho e Silva, IFF} \\ Orientador \\ D.Sc Engenharia de Materiais (UENF)} 
   \assinatura{\textbf{Ronaldo Amaral Santos, IFF} \\ Coorientador \\ M.Sc Pesquisa Operacional e Inteligência Computacional (UCAM)}
   \assinatura{\textbf{Fábio Duncan de Souza, IFF} \\ D.Sc Engenharia e Ciência dos Materiais (UENF)}
   \assinatura{\textbf{Breno Fabricio Terra Azevedo, IFF} \\ D.Sc Informática na Educação (UFRGS)}
   %\assinatura{\textbf{Professor} \\ Convidado 3}
   %\assinatura{\textbf{Professor} \\ Convidado 4}
      
%   \begin{center}
%     \vspace*{0.5cm}
%     {\large\imprimirlocal}
%     \par
%     {\large\imprimirdata}
%     \vspace*{1cm}
%   \end{center}
  
\end{folhadeaprovacao}

% % ---

% % dedicatoria
% \include{secoes/dedicatoria}

% % agradecimentos
% \include{secoes/agradecimentos}

% % epigrafe
% \include{secoes/epigrafe}

% % resumo e abstract
% \setlength{\absparsep}{18pt} % ajusta o espaçamento dos parágrafos do resumo
\begin{resumo}
 
Com o crescente avanço da formação acadêmica e das universidades, os eventos acadêmicos tornam-se cada vez mais frequentes. Em vista disso, o Instituto Federal Fluminense possui o \textit{website} Eventos IFF, no entanto o mesmo não possui uma aplicação móvel. No mercado são ofertadas inúmeras soluções que atendam a demanda de gestão de eventos. No entanto, o desenvolvimento interno de uma solução possibilita uma integração com os serviços internos do Eventos IFF, possibilitando maior customização da aplicação, além de uma maior adequação às características próprias da instituição. Sendo assim, foi decidido que o desenvolvimento de um aplicativo móvel próprio para a plataforma seria mais adequada a necessidade da instituição. O projeto foi estruturada numa metodologia visando obter melhores resultados e validações. Como resultado, foi construído um protótipo interativo estruturando a solução proposta, visando guiar o desenvolvimento do aplicativo, trazendo a visão de como a solução foi desenhada e projetada. A solução se mostrou viável, agregando agilidade e eficiência no processo de gestão de eventos pelo Eventos IFF.


 \textbf{Palavras-chave}: Instituto Federal Fluminense, Eventos IFF, Protótipo
\end{resumo}


% \begin{resumo}[Abstract]
 \begin{otherlanguage*}{english}
  
With the growing advancement of academic formation and universities, academic events become more and more frequent. Given this, the Instituto Federal Fluminense has the Eventos IFF website, however, it does not have a mobile application. The market offers numerous solutions that meet the demand for event management. However, the internal development of a solution enables integration with the internal services of Eventos IFF, enabling greater customization of the application, as well as greater adaptation to the institution's characteristics. Therefore, it was decided that the development of a mobile application for the platform would be more adequate to the institution's needs. The project was structured in a methodology aiming to obtain better results and validations. As a result, an interactive prototype was built structuring the proposed solution, aiming to guide the application development, bringing the vision of how the solution was designed and designed. The solution proved to be viable, adding agility and efficiency in the event management process by IFF Events.
 
   \noindent 
   \textbf{Keywords}:  Instituto Federal Fluminense, Eventos IFF, Prototype.
 \end{otherlanguage*}
\end{resumo}

% ---
% inserir lista de ilustrações
% ---
\pdfbookmark[0]{\listfigurename}{lof}
\listoffigures
\cleardoublepage
% ---

% % ---
% % inserir lista de tabelas
% % ---
\pdfbookmark[0]{\listtablename}{lot}
\listoftables*
\cleardoublepage
% % ---

% ---
% inserir lista de abreviaturas e siglas
% ---
\begin{siglas}
  \item[IFF] Instituto Federal Fluminense
  \item[JSON] JavaScript Object Notation
  \item[IHC] Interação Humano–Computador
\end{siglas}
% ---

% % ---
% % inserir o sumario
% % ---
\pdfbookmark[0]{\contentsname}{toc}
\tableofcontents*
\cleardoublepage
% ---



% ----------------------------------------------------------
% ELEMENTOS TEXTUAIS
% ----------------------------------------------------------
\textual

% ----------------------------------------------------------
% inclusao das secoes do texto
% ----------------------------------------------------------
\chapter{Introdução}

Com o crescente avanço da formação acadêmica e das universidades, os eventos acadêmicos tornam-se cada vez mais frequentes. Esses eventos são de grande importância para a formação, visto que é uma forma do aluno interagir com profissionais da área, além de poder submeter artigos acadêmicos sendo fundamentais para a comunidade científica, agregando mais experiência e conhecimento para a formação do aluno.

Em geral, os eventos acadêmicos são compostos por atividades como: congressos, seminários, cursos, mesas-redondas, entre outros, gerando assim grande fluxo de alunos durante essas atividades, e devido à maioria gerar certificados de participação, a presença do aluno deve ser registrada. Com isso, o gerenciamento do evento sem um software ou aplicação resulta em um evento com mais riscos e gargalos. 

Atualmente, há diversas ferramentas digitais para controle de eventos em geral. Umas delas são as redes sociais, como \textit{Facebook} e \textit{Instagram}, que facilitam a divulgação e comunicação com o público alvo, porém são superficiais quando o assunto é gestão. Contudo, algumas plataformas \textit{web} e móveis foram criadas com esse tipo de finalidade, e oferecem funcionalidades para administração de eventos, como \textit{Doity} e \textit{Eventbrite}. 
Conforme a aumento da demanda de eventos acadêmicos, o IFF se viu necessitado de uma plataforma para centralizar a administração deles, sendo assim foi desenvolvido o \textit{website} Eventos IFF.

\section{Objetivos}
\subsection{Objetivos Gerais}

Desenvolver um protótipo interativo da solução para o gerenciamento de eventos integrada a plataforma \textit{web} IFF Eventos e desenhar a arquitetura de um sistema para gestão de eventos.

\subsection{Objetivos Específicos}

Dentre os objetivos específicos do trabalho de conclusão, destacam-se:
\begin{itemize}
    \item Validação da funcionalidade em cenário real de evento acadêmico;
    \item Analisar o cenário de utilização desse tipo de aplicativo, observando e analisando os requisitos necessários para atender a demanda;
    \item Utilizar técnicas de prototipagem para construir um protótipo interativo e próximo da solução proposta;
    \item Construir diagramas para definição da arquitetura da solução. 
\end{itemize}

\section{Justificativa}

Apesar da existência de aplicações para gestão de eventos, as mesmas têm suas principais funcionalidades monetizadas, tornando custosas suas utilizações. Dessa forma, visando contribuir com o IFF, foi decidido realizar o desenvolvimento de um projeto para uma aplicação focada no gerenciamento de eventos acadêmicos utilizando tecnologia móvel e baseando-se na estrutura de organização da plataforma web Eventos IFF.

Com base nisto espera-se que por meio desta aplicação a gestão dos eventos seja otimizada, minimizando assim os riscos e gargalos e impactando positivamente a comunidade acadêmica.

\section{Organização da monografia}

O conteúdo deste trabalho está dividido em 4 capítulos. No Capítulo 2 é apresentado  uma revisão bibliográfica sobre \textit{Firebase} e o conceito de prototipação. No Capítulo 3 é apresentado a metodologia... [vai ser completado futuramente]. Por fim, no Capítulo 4 é apresentado... [vai ser completado futuramente].

\chapter{Revisão Bibliográfica}

\section{Firebase}

\textit{Firebase} é um produto da \textit{Google} que engloba serviços para realização de  processos internos para algumas plataformas e linguagens de programação. Para aplicações que necessitam, por exemplo, de respostas rápidas para um dado modificado ou acrescentado na base dados, o Firebase se torna ideal para realização desse tipo de serviço \cite{firebase}.

O Firebase é dividido em alguns tipos de serviços, que estão listados e detalhados a seguir.

\begin{itemize}
    \item \textit{Real-time Database}: Serviço de armazenamento em nuvem, onde os dados são armazenados e estruturados em formato JSON. Cada cliente é associado aos seus dados e automaticamente são sincronizados, realizando uma resposta rápida quando algum dado é atualizado \cite{firebase};
    \item \textit{Authentication}: Gerencia as requisições referentes a autenticação, tais como \textit{login}. Permite que o usuário crie perfis de autenticação usando \textit{e-mail} e senha, número de telefone, entre outros \cite{firebase};
    \item \textit{Storage}: Armazena conteúdo gerado pelos clientes, tais como arquivos de mídia, e realiza uma transferência segura desse conteúdo quando solicitado \cite{firebase}. 
\end{itemize}

\section{Prototipação}

No processo de desenvolvimento de um sistema, é fundamental a realização de testes antes da versão ser finalizada. Quase sempre os requisitos definidos no início do projeto não atendem completamente ao usuário final, e uma mudança no \textit{software} após sua finalização pode ser muito custoso \cite{prototipacao}.

O ato da prototipação auxilia a enxergar os requisitos que precisam ser revisados ou até retirados do projeto. A prototipação em um projeto de desenvolvimento é uma ótima forma de comunicação entre o usuário alvo e desenvolvedor, pois provê um \textit{feedback} mais realista sobre o contexto onde o \textit{software} pretende ser utilizado \cite{prototipacao}.

A prototipação em IHC conta com algumas técnicas para a sua execução. Para o desenvolvimento deste trabalho, foi decidido desenvolver um protótipo que oferecesse interatividade, aumentando assim a experiência e imersão durante sua utilização.

\subsection{Figma}

\textit{Figma} é uma ferramenta de \textit{design} colaborativa em nuvem, lançada em 2016, e com funcionamento em navegador, sendo possível utilizar em qualquer sistema operacional que execute um navegador da \textit{web} \cite{nguyen}. Por oferecer uma plataforma colaborativa e em nuvem, o \textit{Figma} permite que desenvolvedores, que estejam distantes geograficamente, possam estar atuando e colaborando no mesmo projeto de prototipação \cite{nascimento}.

Existem outras soluções para prototipação no mercado similares ao \textit{Figma}, como o \textit{Adobe XD}. No entanto, o \textit{Figma} apresenta algumas vantagens que o faz ser escolhido para projetos de prototipação \cite{teplov}. Algumas dessas vantagens são:

\begin{itemize}
    \item Por ser uma aplicação baseada na \textit{web}, é possível de ser executada em mais sistemas operacionais. \textit{Adobe XD} apenas pode ser executado em sistema operacional \textit{Windows} e \textit{Mac} \cite{teplov};
    \item É possível compartilhar mais projetos com a conta grátis. \textit{Adobe XD} permite apenas um projeto compartilhado \cite{teplov}.
\end{itemize}
\chapter{Análise do Problema}

Este capítulo tem como objetivo fazer uma análise e contextualização dos problemas descritos por este trabalho. Na seção 3.1 é abordado sobre o conceito de evento acadêmico e seus desafios atuais. Na seção 3.2 é descrito e comparado algumas soluções presentes no mercado para gestão de eventos. Na seção 3.3 é descrita a solução proposta por esse trabalho para auxílio na gestão de eventos da plataforma Eventos IFF. Por fim, na seção 3.4 é apresentado diagramas descrevendo o cenário do ciclo de vida dos eventos na plataforma Eventos IFF.




\chapter{Metodologia}

Na Figura x é ilustrado em fluxograma as etapas citadas neste capítulo para desenvolvimento do trabalho.

<Colocar um fluxograma das atividades>

Esse trabalho foi desenvolvido seguindo as seguintes etapas, nas respectivas ordens: 

\begin{enumerate}
  \item Desenvolvimento de um aplicativo para validação da funcionalidade de credenciamento;
  \item Testes do aplicativo: nessa etapa é realizado testes executados pelos desenvolvedores para validação;
  \item Execução do aplicativo em ambiente real: o aplicativo é disponibilizado para uso no evento Congresso Integrado da Tecnologia da Informação (CITI) 2019;
  \item Coleta das opiniões dos usuários do aplicativo através de formulário;
  \item Análise do resultado obtido com o formulário de opinião;
  \item Avaliação da viabilidade da ferramenta para credenciamento em eventos acadêmicos;
  \item Estudo da plataforma \textit{web} Eventos IFF;
  \item Levantamento de requisitos para desenvolvimento do aplicativo móvel para o Eventos IFF;
  \item Avaliação da viabilidade técnica de implementação e integração de uma ferramenta móvel para a plataforma Eventos IFF;
  \item Desenho de \textit{wireframes} das funcionalidades levantadas com a análise de requisitos;
  \item Desenho dos diagramas de caso de uso e de classe;
  \item Desenho da arquitetura da solução;
  \item Estudo das ferramentas para prototipação existentes;
  \item Desenvolvimento do protótipo na plataforma selecionada;
  \item Revisão e correção do protótipo;
  \item Termo de aceite do protótipo.
\end{enumerate}
\chapter{Desenho da solução}

Esse capítulo tem como objetivo descrever como foi desenvolvido e desenhado a solução proposta por esse trabalho, além de apresentar resultados obtidos durante o processo.

\section{Solução proposta}

Como destacado por este trabalho anteriormente, há soluções dispostas no mercado que atendem a demanda de gestão de eventos. No entanto, foi identificado que o desenvolvimento de um aplicativo seria uma hipótese, visto os requisitos e anseios da instituição, possibilitando também demais personalizações conforme as necessidades do requisitante.

Dentre todas as citadas na Seção \ref{sec:mercado}, a que mais se aproximou do desejado foi o \textit{Doity}, porém não possui integração com sistemas terceiros, assim inviabilizando uma integração com a plataforma \textit{web} Eventos IFF. Além disso, há limitação na quantidade de participantes em eventos grátis, características as quais são essenciais.

Para a escolha da plataforma, havia como requisito características indispensáveis. A mesma deveria ser gratuita, deve apresentar uma fácil integração com a plataforma Eventos IFF, emissão de certificados e possibilidade de interação entre palestrante e inscritos, sendo assim optou-se pelo desenvolvimento de uma nova ferramenta.

\section{Diagramas da solução}

Na Figura \ref{fig:caso-de-uso} é ilustrado através do diagrama de uso as funcionalidades e ações que o usuário poderá executar no aplicativo Eventos IFF. O diagrama conta com dois atores. A audiência é o usuário que tem papel de participante do evento, já o organizador é o usuário que realiza a administração do evento, tendo acesso a todas as funcionalidades administrativas do aplicativo.

\begin{figure}[H]
    \centering
    \caption{Diagrama de caso de uso}
    \fbox{\includegraphics[scale=0.36]{figuras/caso-de-uso.jpg}}
    \label{fig:caso-de-uso}
    \legend{Fonte: elaborado pelos autores}
\end{figure}


Na Figura \ref{fig:digrama-classe} é apresentado o diagrama de classe conceitual, onde é ilustrado as principais entidades que compõem o aplicativo Eventos IFF e suas relações. Também é apresentado os 4 papeis que o usuário pode ter em relação a um evento, que são: audiência, organizador, palestrante e voluntário. Cada papel possui as funcionalidades que os usuários podem executar no aplicativo.

\begin{figure}[H]
    \centering
    \caption{Diagrama de classe conceitual}
    \fbox{\includegraphics[scale=0.49]{figuras/diagrama-classe-conceitual.jpg}}
    % \includegraphics[scale=0.5]{figuras/diagrama-classe-conceitual.jpg}
    \label{fig:digrama-classe}
    \legend{Fonte: elaborado pelos autores}
\end{figure}

\section{Validação e execução em ambiente real}

Em novembro de 2019, no IFF Campos Centro, ocorreu o evento CITI. Ocorreram nesse evento atividades como palestras, minicursos e mesas redondas, atividades as quais necessitavam registro de presença dos participantes para geração dos certificados. Foi observado, pelos organizadores do evento, a necessidade de uma ferramenta para agilizar o registro dessas presenças. 

Diante disso, foi desenvolvido pelos os autores deste trabalho um aplicativo para dispositivos com sistema operacional \textit{Android} para atender a essa necessidade, o qual também foi utilizado para validação e estudo de parte da solução proposta deste trabalho. O aplicativo foi desenvolvido utilizando a \textit{framework} para desenvolvimento híbrido \textit{Ionic}. No entanto, foi utilizado a técnica de \textit{Minimum Viable Product} (MVP), conforme descrito por \citeonline{ries2014lean}. O objetivo foi construir um aplicativo com recursos e funcionalidades mínimas, porém viáveis, para o processo de registro de presença no evento.

Além do aplicativo, foi desenvolvida uma \textit{Application Programming Interface} (API), na linguagem PHP, que foi hospedada no servidor do IFF Campos Centro, onde possuía apenas uma rota, usando o verbo \textit{Hypertext Transfer Protocol} (HTTP) GET, para registrar a presença em um servidor \textit{Structured Query Language} (SQL). Era armazenado os seguintes dados:

\begin{itemize}
    \item CPF do participante;
    \item Nome do participante;
    \item Identificador da atividade o qual está participando;
    \item Data e hora do registro;
    \item Nome do voluntário que está realizando o registro.
\end{itemize}

O registro era feito na entrada e saída da atividade, e a informação da data e hora do registro era usada para calcular posteriormente o tempo em que o participante ficou na atividade, ajudando na validação para a geração do certificado. Além de registrar utilizando a API, também era gerado um registro dessas presenças em um arquivo \textit{Comma-Separated Values} (CSV) no dispositivo do usuário, com a finalidade de haver um \textit{backup} em caso de perdas no banco de dados.

O aplicativo era composto por duas telas. Na primeira tela, haviam dois campos, um de texto para o voluntário informar seu nome, e outro campo para o voluntário selecionar qual atividade vai ser registrada a presença. Abaixo haviam dois botões, um para prosseguir para a tela de leitura de \textit{QRCode} para registrar a presença do participante. O segundo botão abria o leitor de \textit{QRcode} do dispositivo \textit{Android} para registrar a presença de outro voluntário no dia do evento. Na Figura \ref{fig:mvp1} é apresentada esta tela.

\begin{figure}[H]
    \centering
    \caption{Tela inicial do MVP} 
    % \includegraphics[scale=0.7]{figuras/mvp1.png}
    \fbox{\includegraphics[scale=0.7]{figuras/mvp1.png}}
    \label{fig:mvp1}
    \legend{Fonte: elaborado pelos autores}
\end{figure}

Na segunda tela havia apenas um botão, ao qual abria o leitor de \textit{QRCode} para registrar a presença do participante (Figura \ref{fig:mvp2}). O \textit{QRCode} lido estava nos crachás de identificação dos participantes distribuídos durante o evento. No \textit{QRCode} contido nos crachás, havia a informação do nome do participante e CPF, o qual era capturado pelo aplicativo.

\begin{figure}[H]
    \centering
    \caption{Tela de leitura de \textit{QRCode} de credenciamento do MVP}
    % \includegraphics[scale=0.7]{figuras/mvp2.png}
    \fbox{\includegraphics[scale=0.7]{figuras/mvp2.png}}
    \label{fig:mvp2}
    \legend{Fonte: elaborado pelos autores}
\end{figure}

\section{Análise do resultado da execução em ambiente real}

Esse MVP foi utilizado por 12 discentes que formavam parte do comitê organizador do CITI. Ao final do evento, foi produzido e distribuído pelos autores deste trabalho um questionário avaliativo, utilizando a ferramenta \textit{Google Forms}, para esses discentes. Foram obtidas 11 respostas. Esse questionário teve como finalidade saber a opinião dos usuários quanto a usabilidade e solução proposta pelo aplicativo. O questionário possuía as seguintes perguntas:

\begin{itemize}
    \item A tela de credenciamento é intuitiva?
    \item A tela de leitura do \textit{QRCode} é intuitiva?
    \item De modo geral a aplicação foi de fácil utilização?
    \item O arquivo de \textit{backup} local foi de fácil acesso?
    \item Aplicação apresentou algum tipo de falha?
    \item Caso apresentou falha, descreva.
    \item Sugestões de melhorias.
\end{itemize}

Com as respostas obtidas pelo formulário, foi observado que o \textit{layout} das telas foram bem aceito, com 100\% das respostas indicando que eram intuitivas e de fácil utilização. Não foi indicado nenhuma falha na utilização do aplicativo. Como sugestões, foi informado uma melhoria na funcionalidade de \textit{backup} e colocar mais imagens no aplicativo. O formulário utilizado para a pesquisa de opinião está disponível no Apêndice \ref{apendice1}, assim como o resultado obtido pelo formulário, onde se encontra no Apêndice \ref{apendice2}.

\section{Desenho da estrutura e funcionalidades}

Após análise dos resultados obtidos com a utilização do aplicativo para credenciamento no CITI, foi iniciado a análise da plataforma Eventos IFF, onde foi avaliado as suas funcionalidades e quais funcionalidades poderiam ser acrescentadas com uma aplicação móvel, baseado na experiência obtida no CITI. Foram feitos levantamentos de requisitos para entendimento melhor da plataforma. 

Ao finalizar essa etapa, foi desenhado \textit{wireframes} ilustrando as funcionalidades vistas inicialmente como relevantes para o aplicativo. Na Figura \ref{fig:wireframe1} e \ref{fig:wireframe2} é mostrado dois \textit{wireframes} desenvolvidos nessa etapa, o primeiro ilustrando a página principal do aplicativo e o segundo com a primeira estrutura idealizada para a funcionalidade de envio de perguntas para o palestrante.

% \begin{figure}[H]
%   \begin{minipage}[b]{0.4\textwidth}
%     \caption{Wireframe da tela inicial do aplicativo}
%     \includegraphics[width=\textwidth]{figuras/wireframe1.PNG}
%     \label{fig:wireframe1}
%     \legend{Fonte: elaborado pelos autores}
%   \end{minipage}
%   \hfill
%   \begin{minipage}[b]{0.4\textwidth}
%     \caption{Wireframe da tela de envio de comentários em atividade}
%     \includegraphics[scale=0.37]{figuras/wireframe2.png}
%     \label{fig:wireframe2}
%     \legend{Fonte: elaborado pelos autores}
%   \end{minipage}
% \end{figure}

\begin{figure}[H]
    \centering
    \caption{\textit{Wireframe} da tela inicial do aplicativo}
    % \includegraphics[scale=0.81]{figuras/wireframe1.PNG}
    \fbox{\includegraphics[scale=0.81]{figuras/wireframe1.PNG}}
    \label{fig:wireframe1}
    \legend{Fonte: elaborado pelos autores}
\end{figure}

\begin{figure}[H]
    \centering
    \caption{\textit{Wireframe} da tela de envio de comentários em atividade}
    % \includegraphics[scale=0.4]{figuras/wireframe2.png}
    \fbox{\includegraphics[scale=0.4]{figuras/wireframe2.png}}
    \label{fig:wireframe2}
    \legend{Fonte: elaborado pelos autores}
\end{figure}

Com a definição das funcionalidades e requisitos, foram desenvolvidos os diagramas de caso de uso e de classe. Com esses diagramas, alguns impedimentos e correções nas funcionalidades inicialmente pensadas ficaram mais esclarecidas, o que foram corrigidas com novas revisões nos requisitos levantados e correções nos diagramas. Em seguida foi desenhada a arquitetura dessa solução, visando ilustrar como seria a comunicação entre os serviços do Eventos IFF com o \textit{website} Eventos IFF e o aplicativo proposto por esse trabalho, além da comunicação com a plataforma \textit{Firebase}.

Nas Figuras \ref{fig:classe} é apresentado o diagrama de classe da aplicação. Na Figura \ref{fig:arquitetura} é ilustrada a arquitetura da integração do aplicativo com os serviços do Eventos IFF, assim como com o \textit{Firebase}.

\begin{figure}[H]
    \centering
    \caption{Diagrama de classe}
    % \includegraphics[scale=0.4]{figuras/Diagrama-de-classe.jpg}
    \fbox{\includegraphics[scale=0.365]{figuras/Diagrama-de-classe.jpg}}
    \label{fig:classe}
    \legend{Fonte: elaborado pelos autores}
\end{figure}

\begin{figure}[H]
    \centering
    \caption{Arquitetura da integração do aplicativo com o Eventos IFF}
    % \includegraphics[scale=0.475]{figuras/Arquitetura.png}
    \fbox{\includegraphics[scale=0.445]{figuras/Arquitetura.png}}
    \label{fig:arquitetura}
    \legend{Fonte: elaborado pelos autores}
\end{figure}

\subsection{Uso do \textit{Firebase} na arquitetura}

O \textit{Firebase} possui uma ferramenta, no qual é chamada de \textit{Firestore Database} ou \textit{Cloud Firestore}, onde consiste em um banco de dados não relacional, flexível e escalonável, que é indicado para desenvolvimento de dispositivos móveis, além de aplicações \textit{web} \cite{google_firebase}.

O \textit{Firestore Database} oferece uma sincronia em tempo real entre o dispositivo cliente e o banco de dados. Isso é possível através de ouvintes que atuam no dispositivo cliente através da integração com o \textit{Firestore Database}, onde atualiza os dados no dispositivo cliente logo após sua mudança no banco de dados \cite{google_firebase}. Na Figura \ref{fig:firebase} é exibido um exemplo da estrutura de armazenamento do \textit{Firestore Database} na interface do \textit{Firebase}.

\begin{figure}[H]
    \centering
    \caption{Interface do \textit{Firestore Database} no \textit{Firebase}}
    % \includegraphics[scale=0.5]{figuras/firebase.PNG}
    \fbox{\includegraphics[scale=0.503]{figuras/firebase.PNG}}
    \label{fig:firebase}
    \legend{Fonte: \textit{website} do \textit{Firebase}}
\end{figure}

Deste modo, para a funcionalidade de comentários na atividade do aplicativo Eventos IFF, essa sincronia em tempo real do \textit{Firestore Database} se torna fundamental para uma boa experiência do usuário, onde os novos comentários feitos por outros usuários na atividade são exibidos em tempo real para o usuário que estiver visualizando essa tela.
\chapter{Prototipação}

Neste capítulo será abordado o processo de desenvolvimento do protótipo da solução, mostrando as soluções tomadas e descrições das telas do aplicativo, assim como suas funcionalidades.

\section{Escolha da ferramenta para prototipação}

O processo para desenvolvimento do protótipo interativo foi iniciado após as definições da arquitetura e desenho dos diagramas citados anteriormente. No início foram feitas pesquisas e análises das melhores ferramentas para prototipagem. Os critérios iniciais buscados para a ferramenta de prototipagem foram: interface amigável, funcionalidades que possibilitem criar um protótipo interativo, compartilhamento de projeto em nuvem, liberdade na criação de \textit{layouts} e customização de componentes de tela, além de não ter custo em sua utilização.

Foram identificadas duas ferramentas que atendem esses requisitos, sendo \textit{Adobe XD} e \textit{Figma}. Foi iniciado o desenvolvimento do protótipo no \textit{Adobe XD}. O \textit{Adobe XD} atendeu a expectativa para a criação das telas interativas do protótipo, contudo, após uma nova avaliação, foi visto que o \textit{Figma} possui a possibilidade de duas ou mais pessoas atuarem de forma síncrona no mesmo projeto através do navegador, possibilitando uma maior agilidade para os autores deste trabalho prosseguirem com o desenvolvimento do protótipo.

Deste modo, foi feita a migração do projeto do \textit{Adobe XD} para o \textit{Figma}, o que não apresentou problemas visto que o \textit{Figma} tem compatibilidade com arquivos com a extensão do \textit{Adobe XD}. A partir desse ponto foi continuado o desenvolvimento do protótipo, com reuniões para revisão e aplicação de melhorias necessárias, além de realização de \textit{backup} do projeto para controle de versão. Na Figura \ref{fig:figma} é apresentado a interface gráfica do \textit{Figma}.

\begin{figure}[H]
    \centering
    \caption{Interface do Figma}
    \includegraphics[scale=0.44]{figuras/figma.PNG}
    \label{fig:figma}
    \legend{Fonte: \textit{website} do \textit{Figma}}
\end{figure}

\section{Perfis de usuário do aplicativo Eventos IFF}

O aplicativo Eventos IFF possui 4 perfis de usuário relacionados a cada evento, sendo assim, para cada evento o usuário pode assumir um perfil diferente, dependendo da sua interação com o evento na plataforma. Ao todo são 4 perfis, os quais são listados a seguir:

\begin{itemize}
    \item Audiência: usuário que realiza a inscrição no evento, atuando como participante do mesmo e de suas atividades;
    \item Organizador: é o criador do evento. Esse usuário pode realizar todas as ações relacionadas a gestão de evento no aplicativo;
    \item Voluntário: é o usuário convidado pelo o organizador para atuar na gestão do evento através do aplicativo, no entanto com limitação de ações;
    \item Palestrante: também é um usuário convidado pelo organizador para atuar na gestão do evento, porém ele tem relação apenas com a(s) atividade(s) a qual ele foi convidado para atuar, limitando-se apenas a realizar ações nessa(s) atividade(s).
\end{itemize}

Cada perfil possui as funcionalidades que podem ser executadas no determinado evento, baseando-se em suas respectivas responsabilidades no respectivo evento. Na Tabela \ref{tab:funcionalidades} é relacionado as funcionalidades do aplicativo com os perfis de usuários.

\begin{table}[]
\caption{Relação das funcionalidades e perfis do aplicativo Eventos IFF}
\label{tab:funcionalidades}
\begin{tabular}{|l|l|c|l|l|}
\hline
 & \multicolumn{1}{c|}{Palestrante} & Organizador & \multicolumn{1}{c|}{Voluntário} & \multicolumn{1}{c|}{Audiência} \\ \hline
Registrar presença & \multicolumn{1}{c|}{x} & x & \multicolumn{1}{c|}{x} &  \\ \hline
Criar atividade &  & x &  &  \\ \hline
Editar atividade &  & x & \multicolumn{1}{c|}{x} &  \\ \hline
Excluir atividade &  & x &  &  \\ \hline
Ativar evento &  & x &  &  \\ \hline
Editar evento &  & x &  &  \\ \hline
Convidar usuário para gestão &  & x &  &  \\ \hline
Criar comentário &  & \multicolumn{1}{l|}{} &  & \multicolumn{1}{c|}{x} \\ \hline
Denunciar comentário &  & \multicolumn{1}{l|}{} &  & \multicolumn{1}{c|}{x} \\ \hline
Avaliar comentário &  & \multicolumn{1}{l|}{} &  & \multicolumn{1}{c|}{x} \\ \hline
Excluir comentário & \multicolumn{1}{c|}{x} & x & \multicolumn{1}{c|}{x} & \multicolumn{1}{c|}{x} \\ \hline
Inscrever em evento &  & \multicolumn{1}{l|}{} &  & \multicolumn{1}{c|}{x} \\ \hline
Solicitar emissão de certificado &  & x &  & \multicolumn{1}{c|}{x} \\ \hline
\end{tabular}
\legend{\newline Fonte: elaborado pelos autores}
\end{table}

\section{Jornada do usuário com perfil audiência}

Ao iniciar o aplicativo é necessário realizar a autenticação através da tela de \textit{login} (Figura \ref{fig:audiencia1}), possibilitando também criar cadastro e alterar senha caso o usuário não se lembre da senha. As ações de cadastro e redefinição de senha redirecionam o usuário para o \textit{website} Eventos IFF.

\begin{figure}[H]
    \centering
    \caption{Tela de \textit{login}}
    \includegraphics[scale=0.44]{figuras/Audiencia/1-Login.jpg}
    \label{fig:audiencia1}
    \legend{Fonte: elaborado pelos autores}
\end{figure}

Após a autenticação, é aberto como tela inicial a exibição dos eventos em cartões contendo as imagens de capa do evento, dispostas com efeito carrossel, dividindo os eventos em andamento, eventos classificados como favoritos pelo usuário e eventos já finalizados (Figura \ref{fig:audiencia2}). Além disso, há um campo textual localizado no topo da tela que possibilita o usuário a buscar o evento desejado pelo nome.

\begin{figure}[H]
    \centering
    \caption{Tela inicial}
    \includegraphics[scale=0.44]{figuras/Audiencia/2-TelaPrincipal.jpg}
    \label{fig:audiencia2}
    \legend{Fonte: elaborado pelos autores}
\end{figure}

Ao lado esquerdo da tela inicial, no canto superior, há um ícone relacionado ao menu lateral de navegação, acessando o mesmo será feita a abertura do menu lateral (Figura \ref{fig:audiencia3}), onde possui três opções: eventos, inscrições e agenda. Essas opções tem como respectivas ações:

\begin{itemize}
    \item Eventos: redireciona o usuário para a tela inicial do aplicativo, a qual é descrita anteriormente;
    \item Inscrições: abre uma tela, com a mesma estrutura da interface da tela inicial, no entanto apenas os eventos onde o usuário é inscrito é apresentado;
    \item Agenda: apresenta a agenda das atividades dos eventos onde o usuário está inscrito, separados por dia e hora.
\end{itemize}

\begin{figure}[H]
    \centering
    \caption{Menu lateral esquerdo}
    \includegraphics[scale=0.44]{figuras/Audiencia/3-MenuEsquerdo.jpg}
    \label{fig:audiencia3}
    \legend{Fonte: elaborado pelos autores}
\end{figure}

Já ao lado direito, no canto superior da tela, há o ícone exibindo a silhueta de um usuário, a qual sendo selecionada abre o menu lateral de perfil (Figura \ref{fig:audiencia4}.B), que contém opções que possibilitam a edição de dados pessoais, configurações do aplicativo e possíveis notificações direcionadas ao usuário. Essas notificações são relacionadas aos eventos onde o usuário está inscrito (Figura \ref{fig:audiencia4}.A), com isso, quando houver alterações em alguma atividade onde usuário confirmou presença ou faltam 15 minutos para a atividade ser iniciada, uma notificação é enviada para o usuário e listada nessa tela de notificações. Quando houver alguma notificação nova não lida pelo usuário, um círculo vermelho deve ficar ao lado direito da opção "Notificações", como ilustrado na Figura \ref{fig:audiencia4}.B, sendo que após o usuário abrir a tela de notificações, este círculo deve desaparecer.

\begin{figure}[H]
    \centering
    \caption{(A) Tela de notificações. (B) Menu lateral direito.}
    \includegraphics[scale=0.63]{figuras/Audiencia/4-5.jpg}
    \label{fig:audiencia4}
    \legend{Fonte: elaborado pelos autores}
\end{figure}

Para acessar algum evento, o usuário pode selecionar o cartão desejado na tela inicial (Figura \ref{fig:audiencia1}), onde exibirá em uma nova tela os detalhes a respeito do evento (Figura \ref{fig:audiencia6}). Nesta tela está disposta a capa do evento, nome do evento e uma descrição. Se o usuário não estiver inscrito no evento, abaixo da capa estarão disponíveis três botões. Um botão para realizar a inscrição no evento, um botão com um ícone de calendário para abrir as atividades daquele evento, e um botão com um ícone de compartilhamento que permite o usuário compartilhar o evento em outros aplicativos instalados no seu dispositivo móvel.

\begin{figure}[H]
    \centering
    \caption{Tela de inscrição no evento quando o usuário não está inscrito}
    \includegraphics[scale=0.44]{figuras/Audiencia/6-Evento.jpg}
    \label{fig:audiencia6}
    \legend{Fonte: elaborado pelos autores}
\end{figure}

Ao selecionar o botão de inscrição, o status do usuário muda para inscrito no evento, mudando a aparência e texto do botão, indicando que a ação foi realizada com sucesso (Figura \ref{fig:audiencia7}). Além disso, mais dois botões se tornam visíveis abaixo da capa. Um botão com um ícone de \textit{QRCode}, que possibilita ao usuário visualizar o \textit{QRCode} com as credenciais para o mesmo registrar a presença no evento, e outro botão que permite o usuário registrar o evento na lista de eventos favoritos.

\begin{figure}[H]
    \centering
    \caption{Tela de inscrição no evento com a inscrição realizada}
    \includegraphics[scale=0.44]{figuras/Audiencia/7-EventoInscrito.jpg}
    \label{fig:audiencia7}
    \legend{Fonte: elaborado pelos autores}
\end{figure}
   
O botão com o ícone de calendário, ao ser acessado, exibe as atividades existentes no evento em uma nova tela (Figura \ref{fig:audiencia8}). As atividades são dispostas em formato lista e divididas por dia, em cada data da lista existe um botão com ícone de calendário, que possibilita assim ao usuário marcar na agenda do dispositivo móvel utilizado as atividades desejadas, tendo como função avisar previamente antes da atividade iniciar.

\begin{figure}[H]
    \centering
    \caption{Tela das atividades do evento}
    \includegraphics[scale=0.44]{figuras/Audiencia/8-ListaAtividades.jpg}
    \label{fig:audiencia8}
    \legend{Fonte: elaborado pelos autores}
\end{figure}

O usuário poderá acessar a tela da atividade que deseja visualizar selecionando atividade na lista, nessa tela o mesmo poderá efetuar a inscrição na atividade desejada através do botão “Participar” (Figura \ref{fig:audiencia9}). Nesta mesma tela é apresentado informações sobre a respectiva atividade, como nome, data de realização, descrição e endereço ou localização da atividade.

\begin{figure}[H]
    \centering
    \caption{Tela da atividade selecionada}
    \includegraphics[scale=0.44]{figuras/Audiencia/9-Ativade.jpg}
    \label{fig:audiencia9}
    \legend{Fonte: elaborado pelos autores}
\end{figure}

O usuário pode realizar a ação de enviar um comentário para atividade inscrita, através do botão “Comentar” nesta tela. Após selecionar esse botão, o usuário é redirecionado para a tela de comentários da atividade, onde é exibido os comentários registrados pelos participantes da atividade (Figura \ref{fig:audiencia10}). Cada comentário tem três opções no lado direito, sendo um ícone de avaliação positiva, um ícone de avaliação negativa, e um ícone para denunciar o comentário.

\begin{figure}[H]
    \centering
    \caption{Tela de comentários da atividade}
    \includegraphics[scale=0.44]{figuras/Audiencia/10-Comentarios.jpg}
    \label{fig:audiencia10}
    \legend{Fonte: elaborado pelos autores}
\end{figure}

Os ícones de avaliação tem o objetivo de ajudar na ordenação da lista de comentários, tendo um aspecto colaborativo entre os participantes da atividade. A avaliação positiva atribui 1 ponto positivo no comentário, e a avaliação negativa atribui 1 ponto negativo. Os comentários são ordenados de forma decrescente pela a diferença desses pontos, sendo o comentário com a maior diferença estando no topo da lista, dando assim maior destaque aos comentários com melhor avaliação.

O ícone para denunciar o comentário emite uma notificação para os usuários organizador, voluntário e palestrante, sendo estes tendo a possibilidade de excluir o comentário. Ao selecionar o ícone para denuncia, o aplicativo deve apresentar uma caixa de dialogo para confirmar a ação, assim como apresentado na Figura \ref{fig:audiencia11}.

\begin{figure}[H]
    \centering
    \caption{Mensagem de confirmação da denuncia de comentário}
    \includegraphics[scale=0.44]{figuras/Audiencia/11-Denunciar.jpg}
    \label{fig:audiencia11}
    \legend{Fonte: elaborado pelos autores}
\end{figure}

Na parte inferior dessa tela há dois botões. O primeiro direciona o usuário para uma tela onde o mesmo pode realizar um comentário novo (Figura \ref{fig:audiencia13}). O segundo botão permite ao usuário visualizar seus comentários realizados naquela atividade, onde o mesmo poderá executar a ação de excluir um ou mais desses comentários.

\begin{figure}[H]
    \centering
    \caption{Tela de envio de um novo comentário}
    \includegraphics[scale=0.44]{figuras/Audiencia/13-EnvioComentario.jpg}
    \label{fig:audiencia13}
    \legend{Fonte: elaborado pelos autores}
\end{figure}

\section{Jornada do usuário com perfil de gestão do evento}

O fluxo inicial para os usuários com papel administrativo é o mesmo descrito na seção anterior. Ao se autenticar, o mesmo é direcionado para a tela inicial com a mesma estrutura da Figura \ref{fig:audiencia2}. Também é possível acessar os mesmos menus laterais, como na Figura \ref{fig:audiencia3} e Figura \ref{fig:audiencia4}.B, no entanto, caso o usuário tenha vinculado a ele algum evento, onde o mesmo seja organizador, palestrante ou voluntário, será apresentada uma opção adicional no menu lateral esquerdo, sendo este a opção “Meus Eventos”, como apresentado na Figura \ref{fig:gestao1}. Essa opção redireciona o usuário para a tela dos seus eventos.

\begin{figure}[H]
    \centering
    \caption{Menu lateral esquerdo com a opção "Meus Eventos"}
    \includegraphics[scale=0.44]{figuras/Gestao/1-MenuEsquerdo.jpg}
    \label{fig:gestao1}
    \legend{Fonte: elaborado pelos autores}
\end{figure}

Na tela dos eventos do usuário (Figura \ref{fig:gestao2}.A), o mesmo tem a possibilidade de acompanhar os eventos onde ele tem algum papel administrativo, além de orientações de como solicitar a criação de um novo evento através do botão “Criar Evento”. Ao selecionar este botão, uma caixa de diálogo é apresentada indicando que o usuário poderá criar o evento através do Sistema Unificado de Administração Pública (SUAP) (Figura \ref{fig:gestao2}.B). Selecionando o botão “Ir para o SUAP” nessa caixa de diálogo, o usuário é redirecionado para o \textit{website} do SUAP.

\begin{figure}[H]
    \centering
    \caption{(A) Tela de eventos do usuário. (B) Orientação para criação de evento.}
    \includegraphics[scale=0.63]{figuras/Gestao/2-3.jpg}
    \label{fig:gestao2}
    \legend{Fonte: elaborado pelos autores}
\end{figure}

Na mesma tela (Figura \ref{fig:gestao2}.A), é listada os eventos em formato de cartões, onde nos cartões tem a imagem da capa do evento. Selecionando o cartão do evento, o usuário pode se deparar com duas possibilidades. Caso o evento selecionado já esteja ativado, o usuário é redirecionado para a tela de gestão do evento, a qual será detalhada posteriormente.

Entretanto, caso o evento não esteja ativado ainda, o usuário é levado para uma tela onde o mesmo poderá acompanhar o status de aprovação do evento. Isso é devido ao fato de, quando um evento é criado no SUAP, esse evento necessita de uma aprovação para estar apto para estar visível no Eventos IFF. Nessa etapa o evento pode apresentar três status em relação a essa aprovação, sendo pendente, reprovado ou aprovado. A mensagem na tela varia para cada um destes status, assim como ilustrado na Figura \ref{fig:gestao5}.

\begin{figure}[H]
    \centering
    \caption{Variação da tela de acompanhamento do status de aprovação do evento}
    \includegraphics[scale=0.47]{figuras/Gestao/5-6-7.jpg}
    \label{fig:gestao5}
    \legend{Fonte: elaborado pelos autores}
\end{figure}

Como apresentado na Figura  \ref{fig:gestao5}, caso o evento esteja aprovado pelo SUAP e ainda não esteja ativado, o usuário poderá realizar essa ação pelo botão ‘Ativar evento’. Esta funcionalidade faz com que o evento torna-se ativo e visível no Eventos IFF.

Após o processo de ativação do evento, haverá o redirecionamento para a tela de gestão do evento (Figura \ref{fig:gestao8}). Na tela de gestão do evento, o usuário terá disponível as abas ‘Editar Dados’ e ‘Gerenciar Atividades’. Em ‘Editar Dados’, o usuário poderá editar o nome e descrição do evento, além de ter as opções para gerenciar as atividades e convidar voluntários. Também é possível conferir dados como o status do evento e o total de inscritos no momento.

\begin{figure}[H]
    \centering
    \caption{Tela de gestão do evento}
    \includegraphics[scale=0.44]{figuras/Gestao/8-EditarEvento.jpg}
    \label{fig:gestao8}
    \legend{Fonte: elaborado pelos autores}
\end{figure}

Para inclusão de atividades, o usuário deverá acessar o botão ‘Adicionar’ na seção atividades, após isto o usuário será redirecionado à uma tela de criação de atividade (Figura \ref{fig:gestao9}.A). Nesta tela o usuário terá duas abas, sendo estas ‘Editar Dados’ e ‘Gerenciar Palestrante’, onde na primeira aba será feito o preenchimento de informações como nome e descrição da atividade, data, hora, local e capacidade máxima de participantes permitida na atividade. Caso o usuário queira convidar palestrantes para a atividade, o usuário deverá acessar o botão ‘Adicionar’, localizado na seção palestrantes. Após selecionar este botão, uma caixa de diálogo é aberta onde possibilita o usuário informar o \textit{e-mail} ao qual deseja enviar o convite para se tornar palestrante da atividade (Figura \ref{fig:gestao9}.B). Esse convite deve conter um texto explicativo e um \textit{link} para o usuário aceitar o convite.

\begin{figure}[H]
    \centering
    \caption{(A) Tela de criação de atividade. (B) Orientação para convidar palestrante.}
    \includegraphics[scale=0.63]{figuras/Gestao/9-10.jpg}
    \label{fig:gestao9}
    \legend{Fonte: elaborado pelos autores}
\end{figure}

Na segunda aba desta mesma tela, a aba ‘Gerenciar Palestrante’, é possível realizar a gestão dos palestrantes relacionados à atividade, dando como possibilidade a exclusão dos palestrantes (Figura \ref{fig:gestao11}).

\begin{figure}[H]
    \centering
    \caption{Tela de gestão de palestrantes da atividade}
    \includegraphics[scale=0.44]{figuras/Gestao/11-GestaoPalestrantes.jpg}
    \label{fig:gestao11}
    \legend{Fonte: elaborado pelos autores}
\end{figure}

Finalizando a criação da atividade na aba ‘Editar Dados’ citada anteriormente (Figura \ref{fig:gestao9}.A), o usuário deverá selecionar o botão "Salvar" \hspace{1mm} localizado na parte inferior da tela para disponibilizar a atividade no evento. Dessa forma ele será redirecionado à tela de gestão do evento (Figura \ref{fig:gestao8}).

Para inclusão de voluntários, o usuário deverá acessar o botão ‘Adicionar’ na seção voluntários, sendo assim levado para a tela de gestão de voluntários do evento (Figura \ref{fig:gestao12}.A). Nesta tela, o usuário terá a visibilidade de todos os usuários voluntários atualmente vinculados ao evento, possibilitando remover alguém. Para convidar um novo voluntário, basta selecionar o botão ‘Adicionar’. Essa ação irá abrir uma caixa de diálogo seguindo o mesmo padrão da ação de convite de palestrante, assim como na Figura \ref{fig:gestao12}.B. Após isso o usuário deve preencher no campo de texto exibido o \textit{e-mail} ao qual deseja enviar o convite. Este convite deve seguir o mesmo padrão do convite do palestrante citado anteriormente.

\begin{figure}[H]
    \centering
    \caption{(A) Tela de gestão de voluntários. (B) Orientação para convidar voluntário.}
    \includegraphics[scale=0.63]{figuras/Gestao/12-13.jpg}
    \label{fig:gestao12}
    \legend{Fonte: elaborado pelos autores}
\end{figure}

	
Selecionando o botão “Voltar” nesta tela (Figura \ref{fig:gestao12}.A) o usuário retorna a tela de gestão do evento, como na Figura \ref{fig:gestao8}. O usuário poderá realizar a gestão das atividades na aba ‘Gerenciar Atividades’ desta tela. Nesta aba o usuário terá a visão das atividades cadastradas no evento e poderá editá-las selecionando a atividade desejada (Figura \ref{fig:gestao14}).

\begin{figure}[H]
    \centering
    \caption{Tela de gestão das atividades do evento}
    \includegraphics[scale=0.44]{figuras/Gestao/14-GerenciarAtividades.jpg}
    \label{fig:gestao14}
    \legend{Fonte: elaborado pelos autores}
\end{figure}

Ao acessar a atividade desejada o usuário terá na tela as informações a respeito da mesma, como status da atividade, capacidade máxima de participantes e data e hora do início (Figura \ref{fig:gestao15}). Além disso, o usuário poderá editar ou excluir a atividade selecionando os botões na parte superior da página. A ação de edição redireciona para a mesma tela de criação de atividade, porém com os campos de texto e demais informações já preenchidos. O usuário poderá realizar as ações de credenciamento por esta tela (Figura \ref{fig:gestao15}) através de duas possibilidades. A primeira possibilidade é por meio do campo de texto, onde usuário informa a credencial do participante e seleciona o botão “Registrar presença” para concluir a ação.

\begin{figure}[H]
    \centering
    \caption{Tela da atividade selecionada}
    \includegraphics[scale=0.44]{figuras/Gestao/15-Atividade.jpg}
    \label{fig:gestao15}
    \legend{Fonte: elaborado pelos autores}
\end{figure}

A segunda possibilidade se dá por meio do botão “Ler \textit{QRCode}”, onde ao selecionar esse botão, o aplicativo deve abrir o leitor de \textit{QRCode} do dispositivo móvel, possibilitando realizar a leitura do \textit{QRCode} de credenciamento do participante.

Logo abaixo desta mesma tela, o usuário poderá ver a lista de participantes que já realizaram o credenciamento, assim como conseguir a informação de quantos participantes faltam realizar o credenciamento dos que confirmaram a presença anteriormente.

Para acessar os comentários do evento, o usuário deverá acessar a aba ‘Comentários’ desta tela. Nessa aba, o usuário terá a possibilidade de ler os comentários feitos pelos os participantes e realizar a gestão dos mesmos (Figura \ref{fig:gestao17}.A). Os comentários são ordenados pelo critério de avaliação citado anteriormente. Caso seja necessário excluir algum comentário, pelo o motivo, por exemplo, de o mesmo ter sido denunciado, o usuário poderá realizar essa ação deslizando o comentário para a esquerda. Isso irá habilitar o botão de exclusão do comentário para completar a ação (Figura \ref{fig:gestao17}.B).

\begin{figure}[H]
    \centering
    \caption{(A) Tela de gestão dos comentários. (B) Ação de exclusão do comentário.}
    \includegraphics[scale=0.63]{figuras/Gestao/17-18.jpg}
    \label{fig:gestao17}
    \legend{Fonte: elaborado pelos autores}
\end{figure}

Finalizando, é importante destacar que essa jornada da gestão de evento descrita por essa seção pode ter como três tipos de usuário vinculados por evento, sendo estes organizador, palestrante e voluntário. O usuário organizador consegue executar todas as funcionalidades descritas nesta jornada, sendo o usuário criador do evento já atribuído automaticamente como organizador. 

O usuário voluntário se limita apenas às ações de realizar credenciamento em atividade, editar atividade e excluir comentários. O usuário palestrante pode apenas executar ações na atividade a qual está vinculado. Essas ações são realizar credenciamento em atividade e excluir comentários.

\chapter{Conclusão}

No processo de desenvolvimento deste trabalho foram seguidas as etapas de análise, estudo e levantamento de requisitos, que são fundamentais para o desenho e desenvolvimento de \textit{softwares}.

Com a execução da técnica de MVP em um evento acadêmico real, foi observado as reais necessidades para esse cenário, e utilizando a pesquisa de opinião da utilização desse MVP possibilitou um melhor esclarecimento da solução desejada.

Deste modo, foi possível realizar os levantamentos com um especialista da plataforma IFF Eventos, e o mesmo considerou satisfatório os resultados obtidos. A partir disso foi constatada a viabilidade do desenvolvimento de uma solução para uma ferramenta que agregasse funcionalidade e agilidade no gerenciamento de eventos da plataforma.

O desenho de diagramas de uso, diagrama de classe conceitual e diagramas de arquitetura foram fundamentais para prosseguir com o desenvolvimento do protótipo de forma mais segura e assertiva, visto que esses diagramas possibilitam um entendimento mais claro do \textit{software} e seu contexto, evitando futuros imprevistos no desenvolvimento.

Paralelamente, foi visto que a técnica de prototipagem auxilia no processo de desenvolvimento de \textit{software}. A construção de um protótipo interativo coopera na visualização da solução, trazendo assim mais facilidade para o desenvolvimento das telas do \textit{software}, uma vez que a etapa visual já está desenhada. 

Por fim, a partir desse desenho de solução, experimentado a partir deste protótipo, foi possível direcionar o sistema, melhorando assim os processos de gerenciamento, bem como otimizando a experiência da audiência nos eventos.



% ----------------------------------------------------------
% ELEMENTOS PÓS-TEXTUAIS
% ----------------------------------------------------------
\postextual
% ----------------------------------------------------------

% ----------------------------------------------------------
% Referências bibliográficas
% ----------------------------------------------------------
\bibliography{referencias.bib}


%---------------------------------------------------------------------
% INDICE REMISSIVO
%---------------------------------------------------------------------
\phantompart
\printindex
%---------------------------------------------------------------------
\begin{apendicesenv}
% Imprime uma página indicando o início dos apêndices
\partapendices
% ----------------------------------------------------------
% Incluir Apêndice
% ----------------------------------------------------------
% ----------------------------------------------------------
% Capitulo com exemplos de comandos inseridos de arquivo externo 
% ----------------------------------------------------------
\chapter{Formulário para pesquisa de opinião sobre o MVP utilizado no CITI 2019}\label{apendice1}

\begin{figure}[H]
    \centering
    \includegraphics[scale=0.35]{figuras/questionario1.jpg}
\end{figure}

\begin{figure}[H]
    \centering
    \includegraphics[scale=0.35]{figuras/questionario2.jpg}
\end{figure}

\begin{figure}[H]
    \centering
    \includegraphics[scale=0.35]{figuras/questionario2.jpg}
\end{figure}

\chapter{Resultado da pesquisa de opinião sobre o MVP utilizado no CITI 2019}\label{apendice2}

\begin{figure}[H]
    \centering
    \includegraphics[scale=0.55]{figuras/form1.png}
\end{figure}

\begin{figure}[H]
    \centering
    \includegraphics[scale=0.55]{figuras/form2.png}
\end{figure}

\begin{figure}[H]
    \centering
    \includegraphics[scale=0.55]{figuras/form3.png}
\end{figure}

\begin{figure}[H]
    \centering
    \includegraphics[scale=0.55]{figuras/form4.png}
\end{figure}

\begin{figure}[H]
    \centering
    \includegraphics[scale=0.55]{figuras/form5.png}
\end{figure}

\begin{figure}[H]
    \centering
    \includegraphics[scale=0.55]{figuras/form6.png}
\end{figure}

\end{apendicesenv}

\end{document}
