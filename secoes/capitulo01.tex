\chapter{Introdução}

Com o crescente avanço da formação acadêmica e das universidades, os eventos acadêmicos tornam-se cada vez mais frequentes. Esses eventos são de grande importância para a formação, visto que é uma forma do aluno interagir com profissionais da área, além de poder submeter artigos acadêmicos sendo fundamentais para a comunidade científica, agregando mais experiência e conhecimento para a formação do aluno.

Em geral, os eventos acadêmicos são compostos por atividades como: congressos, seminários, cursos, mesas-redondas, entre outros, gerando assim grande fluxo de alunos durante essas atividades, e devido à maioria gerar certificados de participação, a presença do aluno deve ser registrada. Com isso, o gerenciamento do evento sem um software ou aplicação resulta em um evento com mais riscos e gargalos. 

Atualmente, há diversas ferramentas digitais para controle de eventos em geral. Umas delas são as redes sociais, como \textit{Facebook} e \textit{Instagram}, que facilitam a divulgação e comunicação com o público alvo, porém são superficiais quando o assunto é gestão. Contudo, algumas plataformas \textit{web} e móveis foram criadas com esse tipo de finalidade, e oferecem funcionalidades para administração de eventos, como \textit{Doity} e \textit{Eventbrite}. 
Conforme a aumento da demanda de eventos acadêmicos, o IFF se viu necessitado de uma plataforma para centralizar a administração deles, sendo assim foi desenvolvido o \textit{website} Eventos IFF.

\section{Objetivos}
\subsection{Objetivos Gerais}

Desenvolver um protótipo interativo da solução para o gerenciamento de eventos integrada a plataforma \textit{web} IFF Eventos e desenhar a arquitetura de um sistema para gestão de eventos.

\subsection{Objetivos Específicos}

Dentre os objetivos específicos do trabalho de conclusão, destacam-se:
\begin{itemize}
    \item Validação da funcionalidade em cenário real de evento acadêmico;
    \item Analisar o cenário de utilização desse tipo de aplicativo, observando e analisando os requisitos necessários para atender a demanda;
    \item Utilizar técnicas de prototipagem para construir um protótipo interativo e próximo da solução proposta;
    \item Construir diagramas para definição da arquitetura da solução. 
\end{itemize}

\section{Justificativa}

Apesar da existência de aplicações para gestão de eventos, as mesmas têm suas principais funcionalidades monetizadas, tornando custosas suas utilizações. Dessa forma, visando contribuir com o IFF, foi decidido realizar o desenvolvimento de um projeto para uma aplicação focada no gerenciamento de eventos acadêmicos utilizando tecnologia móvel e baseando-se na estrutura de organização da plataforma web Eventos IFF.

Com base nisto espera-se que por meio desta aplicação a gestão dos eventos seja otimizada, minimizando assim os riscos e gargalos e impactando positivamente a comunidade acadêmica.

\section{Organização da monografia}

O conteúdo deste trabalho está dividido em 4 capítulos. No Capítulo 2 é apresentado  uma revisão bibliográfica sobre \textit{Firebase} e o conceito de prototipação. No Capítulo 3 é apresentado a metodologia... [vai ser completado futuramente]. Por fim, no Capítulo 4 é apresentado... [vai ser completado futuramente].
