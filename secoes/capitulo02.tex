\chapter{Revisão Bibliográfica}

\section{Firebase}

\textit{Firebase} é um produto da \textit{Google} que engloba serviços para realização de  processos internos para algumas plataformas e linguagens de programação. Para aplicações que necessitam, por exemplo, de respostas rápidas para um dado modificado ou acrescentado na base dados, o \textit{Firebase} se torna ideal para realização desse tipo de serviço \cite{firebase}.

O \textit{Firebase} é dividido em alguns tipos de serviços, que estão listados e detalhados a seguir.

\begin{itemize}
    \item \textit{Real-time Database}: Serviço de armazenamento em nuvem, onde os dados são armazenados e estruturados em formato \textit{JavaScript Object Notation} (JSON). Cada cliente é associado aos seus dados e automaticamente são sincronizados, realizando uma resposta rápida quando algum dado é atualizado \cite{firebase};
    \item \textit{Authentication}: Gerencia as requisições referentes a autenticação, tais como \textit{login}. Permite que o usuário crie perfis de autenticação usando \textit{e-mail} e senha, número de telefone, entre outros \cite{firebase};
    \item \textit{Storage}: Armazena conteúdo gerado pelos clientes, tais como arquivos de mídia, e realiza uma transferência segura desse conteúdo quando solicitado \cite{firebase}. 
\end{itemize}

\section{Prototipação}

No processo de desenvolvimento de um sistema, é fundamental a realização de testes antes da versão ser finalizada. Quase sempre os requisitos definidos no início do projeto não atendem completamente ao usuário final, e uma mudança no \textit{software} após sua finalização pode ser muito custosa \cite{prototipacao}.

O ato da prototipação auxilia a enxergar os requisitos que precisam ser revisados ou até retirados do projeto. A prototipação em um projeto de desenvolvimento é uma ótima forma de comunicação entre o usuário alvo e desenvolvedor, pois provê um \textit{feedback} mais realista sobre o contexto onde o \textit{software} pretende ser utilizado \cite{prototipacao}.

A prototipação em Interação Humano–Computador (IHC) conta com algumas técnicas para a sua execução. Para o desenvolvimento deste trabalho, foi decidido desenvolver um protótipo que oferecesse interatividade, aumentando assim a experiência e imersão durante sua utilização.

\subsection{Figma}

\textit{Figma} é uma ferramenta de \textit{design} colaborativa em nuvem, lançada em 2016, e com funcionamento em navegador, sendo possível utilizar em qualquer sistema operacional que execute um navegador da \textit{web} \cite{nguyen}. Por oferecer uma plataforma colaborativa e em nuvem, o \textit{Figma} permite que desenvolvedores, que estejam distantes geograficamente, possam estar atuando e colaborando no mesmo projeto de prototipação \cite{nascimento}.

Existem outras soluções para prototipação no mercado similares ao \textit{Figma}, como o \textit{Adobe XD}. No entanto, o \textit{Figma} apresenta algumas vantagens que o faz ser escolhido para projetos de prototipação \cite{teplov}. Algumas dessas vantagens são:

\begin{itemize}
    \item Por ser uma aplicação baseada na \textit{web}, é possível de ser executada em mais sistemas operacionais. \textit{Adobe XD} apenas pode ser executado em sistema operacional \textit{Windows} e \textit{Mac} \cite{teplov};
    \item É possível compartilhar mais projetos com a conta grátis. \textit{Adobe XD} permite apenas um projeto compartilhado \cite{teplov}.
\end{itemize}