\setlength{\absparsep}{18pt} % ajusta o espaçamento dos parágrafos do resumo
\begin{resumo}
 
Com o crescente avanço da formação acadêmica e das universidades, os eventos acadêmicos tornam-se cada vez mais frequentes. Em vista disso, o Instituto Federal Fluminense possui o \textit{website} Eventos IFF, no entanto o mesmo não possui uma aplicação móvel. No mercado são ofertadas inúmeras soluções que atendam a demanda de gestão de eventos. No entanto, o desenvolvimento interno de uma solução possibilita uma integração com os serviços internos do Eventos IFF, possibilitando maior customização da aplicação, além de uma maior adequação às características próprias da instituição. Sendo assim, foi decidido que o desenvolvimento de um aplicativo móvel próprio para a plataforma seria mais adequada a necessidade da instituição. O projeto foi estruturada numa metodologia visando obter melhores resultados e validações. Como resultado, foi construído um protótipo interativo estruturando a solução proposta, visando guiar o desenvolvimento do aplicativo, trazendo a visão de como a solução foi desenhada e projetada. A solução se mostrou viável, agregando agilidade e eficiência no processo de gestão de eventos pelo Eventos IFF.


 \textbf{Palavras-chave}: Instituto Federal Fluminense, Eventos IFF, Protótipo
\end{resumo}

