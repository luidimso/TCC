\chapter{Desenho da solução}

Esse capítulo tem como objetivo descrever como foi desenvolvido e desenhado a solução proposta por esse trabalho, além de apresentar resultados obtidos durante o processo.

\section{Validação e execução em ambiente real}

Em novembro de 2019, no IFF Campos Centro, ocorreu o evento CITI, Congresso Integrado da Tecnologia da Informação. Ocorreram nesse evento atividades como palestras, minicursos e mesas redondas, atividades as quais necessitavam registro de presença dos participantes para geração dos certificados. Foi observado, pelos organizadores do evento, a necessidade de uma ferramenta para agilizar o registro dessas presenças. 

Diante disso, foi desenvolvido pelos os autores deste trabalho um aplicativo para dispositivos com sistema operacional \textit{Android} para atender a essa necessidade, o qual também foi utilizado para validação e estudo de parte da solução proposta deste trabalho. O aplicativo foi desenvolvido utilizando a \textit{framework} para desenvolvimento híbrido \textit{Ionic}. No entanto, foi utilizado a técnica de \textit{Minimum Viable Product} (MVP), conforme descrito por \citeonline{ries2014lean}. O objetivo foi construir um aplicativo com recursos e funcionalidades mínimas, porém viáveis, para o processo de registro de presença no evento.

Além do aplicativo, foi desenvolvida uma API, na linguagem PHP, que foi hospedada no servidor do IFF Campos Centro, onde possuía apenas uma rota, usando o verbo HTTP GET, para registrar a presença em um servidor SQL. Era armazenado os seguintes dados:

\begin{itemize}
    \item CPF do participante;
    \item Nome do participante;
    \item Identificador da atividade o qual está participando;
    \item Data e hora do registro;
    \item Nome do voluntário que está realizando o registro.
\end{itemize}

O registro era feito na entrada e saída da atividade, e a informação da data e hora do registro era usada para calcular posteriormente o tempo em que o participante ficou na atividade, ajudando na validação para a geração do certificado. Além de registrar utilizando a API, também era gerado um registro dessas presenças em um arquivo CSV no dispositivo do usuário, com a finalidade de haver um \textit{backup} em caso de perdas no banco de dados.

O aplicativo era composto por duas telas. Na primeira tela, haviam dois campos, um de texto para o voluntário informar seu nome, e outro campo para o voluntário selecionar qual atividade vai ser registrada a presença. Abaixo haviam dois botões, um para prosseguir para a tela de leitura de \textit{QRCode} para registrar a presença do participante. O segundo botão abria o leitor de \textit{QRcode} do dispositivo \textit{Android} para registrar a presença de outro voluntário no dia do evento.

Na segunda tela havia apenas um botão, ao qual abria o leitor de \textit{QRCode} para registrar a presença do participante. O \textit{QRCode} lido estava nos crachás de identificação dos participantes distribuídos durante o evento. No \textit{QRCode} contido nos crachás, havia a informação do nome do participante e CPF, o qual era capturado pelo aplicativo.

<colocar imagem das telas aqui>

\section{Análise do resultado da execução em ambiente real}

Esse MVP foi utilizado por 12 discentes que formavam parte do comitê organizador do CITI. Ao final do evento, foi produzido e distribuído pelos autores deste trabalho um questionário avaliativo, utilizando a ferramenta \textit{Google Forms}, para esses discentes. Foram obtidas 11 respostas. Esse questionário teve como finalidade saber a opinião dos usuários quanto a usabilidade e solução proposta pelo aplicativo. O questionário possuía as seguintes perguntas:

\begin{itemize}
    \item A tela de credenciamento é intuitiva?
    \item A tela de leitura do \textit{QRCode} é intuitiva?
    \item De modo geral a aplicação foi de fácil utilização?
    \item O arquivo de \textit{backup} local foi de fácil acesso?
    \item Aplicação apresentou algum tipo de falha?
    \item Caso apresentou falha, descreva.
    \item Sugestões de melhorias
\end{itemize}

Com as respostas obtidas pelo formulário, foi observado que o \textit{layout} das telas foram bem aceito, com 100\% das respostas indicando que eram intuitivas e de fácil utilização. Não foi indicado nenhuma falha na utilização do aplicativo. Como sugestões, foi informado uma melhoria na funcionalidade de \textit{backup} e colocar mais imagens no aplicativo.

<colocar imagem dos gráficos aqui>