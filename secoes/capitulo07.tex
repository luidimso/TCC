\chapter{Conclusão}

No processo de desenvolvimento deste trabalho foram seguidas as etapas de análise, estudo e levantamento de requisitos, que são fundamentais para o desenho e desenvolvimento de \textit{softwares}.

Com a execução da técnica de MVP em um evento acadêmico real, foi observado as reais necessidades para esse cenário, e utilizando a pesquisa de opinião da utilização desse MVP possibilitou um melhor esclarecimento da solução desejada.

Deste modo, foi possível realizar os levantamentos com um especialista da plataforma Eventos IFF, e o mesmo considerou satisfatório os resultados obtidos. A partir disso foi constatada a viabilidade do desenvolvimento de uma solução para uma ferramenta que agregasse funcionalidade e agilidade no gerenciamento de eventos da plataforma.

O desenho de diagramas de uso, diagrama de classe conceitual e diagramas de arquitetura foram fundamentais para prosseguir com o desenvolvimento do protótipo de forma mais segura e assertiva, visto que esses diagramas possibilitam um entendimento mais claro do \textit{software} e seu contexto, evitando futuros imprevistos no desenvolvimento.

Paralelamente, foi visto que a técnica de prototipagem auxilia no processo de desenvolvimento de \textit{software}. A construção de um protótipo interativo coopera na visualização da solução, trazendo assim mais facilidade para o desenvolvimento das telas do \textit{software}, uma vez que a etapa visual já está desenhada. 

Por fim, a partir desse desenho de solução, experimentado a partir deste protótipo, foi possível direcionar o sistema, melhorando assim os processos de gerenciamento, bem como otimizando a experiência da audiência nos eventos.
